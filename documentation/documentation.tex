\documentclass{article}
\usepackage[utf8]{inputenc}

\title{OSRT dokumentácia}
\author{Michal Petkáč}

\begin{document}

\maketitle
\section{Funkcia programu}
Program je určený na uchovávanie a spracovanie záznamov o veku a plate.

\section{Požiadavky}
Program je určený pre OS Linux. Testovaný bol na OS Ubuntu 18.04.3 LTS. 

\section{Kompilácia}
Kompilácia prebieha cez program make. Stačí do konzoly zadať príkaz make a program sa skompiluje. Na odstránenie súbrov vytvorených pri kompilácii slúži pravidlo clean vykonané príkazom make clean.

\section{Spustenie}
Program pozostáva z časti servera a časti klienta. Server sa spustí z terminálu príkazom ./server. 
\newline
Klientská časť sa spúšta príkazom ./client. Táto časť má viacero možností spustenia. V prípade spustenia bez argumentov, alebo s argumentom 1 sa spustí jeden klient, ktorý je ovládaný z konzoly. V prípade agrumentov 2-4 sa spustí klient ovládaný z konzoly plus 1-3 automatický klienti, ktorých správanie je definované v súboroch client1 - client3.

\section{Ukončenie}
Pre správne ukončenie klientskej časti stačí zadať pri komunikácii -1, čo ukončí komunikáciu so serverom a zavrie sockety. Ošetrené sú aj prípady, keď je klient ukončený signálmi SIGINT, alebo SIGHUP.
\newline
Pre správne ukončenie činnosti servera treba poslať signál SIGINT. Keď server príjme tento signál, najskôr ukončí všetky spustené serverové, aj klientské procesy, vymaže odstráni zdieľané pamäte a alokované premenné a nakoniec zatvorí socket a vypne sa.

\section{Popis programu}
\subsection{Pripájanie klientov}
Server pri spustení počúva na dopredu stanovenom porte na klientov. Pri pripojení klienta prebehne inicializačná komunikácia. Klient pošle serveru svoj PID. Server náje voľný port, otvorí ho a pošle klientovi číslo tohto portu. Server vytvorí pre klienta obslužný proces, zatvorí socket a čaká na ďalšie pripojenie. Klient následne komunikuje výhradne s novovytvoreným procesom.

\subsection{Komunikácia s klientmi}
Komunikácia medzi serverom a klientov prebieha na základe požiadaviek. Klient pošle serveru číslo požiadavky, ktoré je stanovené v súbore header.h. Server následne vie, ako sa má zachovať. Po spracovaní požiadavky odpovie server, či bola požiadavka správne spracovaná. Ak áno, server pošile klientovi 1, ak nie, tak 0.

\section{Zoznam súborov}
\subsection{Súbory s kódom (.c)}
Súbor server.c obsahuje kód serveru. Súbor client.c obsahuje kód klienta.
\newline
V súbore functions.c sa nachádzajú funkcie využívané aj serverom, ak klientom. Sú to najmä funkcie na prácu so socketmi, zdieľanými pamäťami a časovačmi.
\newline
V súbore statistic.c sa nachádzajú funkcie používané serverom na prácu s databázou. 
\newline
V súbore requests.c sa nachádzajú funkcie používané klientom na prijímanie a zobrazovanie informácií prijatých od servera.
\subsection{Hlavičkové súbory (.h)}
V hlavičkových súboroch, ktoré sú rovnomenné, ako súbory s kódom sa nachádzajú hlavičky funkcií. Je to z toho dôvodu, aby kompilácia každého súboru mohla prebiehať samostatne. Výstupné súbory sa potom zlinkujú a vytvorí sa výsledný súbor. 
\newline
Špeciálnym súborom je súbor header.h, kde sa nachádzajú definície makier, include súbory a štruktúry používané v celom programe.
\subsection{Iné súbory}
Okrem vyššie spomenutých súborov sa používajú aj súbory client1 - client3. Tieto súbory definujú správanie klientov. Použité môžu byť súbory s indexami do 9 (client9), avšak za predpokladu, že sa v súbore header.h zmení konštanta MAX\_BOT\_CLIENTS na požadovanú veľkosť.
\subsection{Stiahnutie}
Kompletný program si môžete stiahnuť z github-u cez nasledovný link:
\newline
https://github.com/pekacc/OSRT-zadanie

\end{document}
